\documentclass[12pt,a4paper]{article}
\usepackage{geometry}
\geometry{left=2.5cm,right=2.5cm,top=2.0cm,bottom=2.5cm}
\usepackage[english]{babel}
\usepackage{amsmath,amsthm}
\usepackage{amsfonts}
\usepackage{bm}
\usepackage[longend,ruled,linesnumbered]{algorithm2e}
\usepackage{siunitx}
\usepackage{fancyhdr}
\usepackage{ctex}
\usepackage{array}
\usepackage{listings}
\usepackage{color}
\usepackage{graphicx}
\usepackage{float}
\usepackage{caption}
\usepackage{longtable}

\begin{document}
    \title{\heiti{《机器学习》课程第 {$4$} 次作业}}
    \date{}
    \author{姓名:\underline{刘哲}~~~~~~学号:\underline{2022103691}~~~~~~}
    \maketitle
    \section{\heiti{第5题}}
    \vspace{10pt}
    \subsubsection*{a}
    岭回归最优化问题的目标函数为
    \begin{equation}
        Obj_{ridge}=\sum_{i=1}^2\left(y_i-\beta_0-\beta_1x_{i1}-\beta_2x_{i2}\right)^2+\lambda\left(\beta_1^2+\beta_2^2\right)
    \end{equation}
    \subsubsection*{b}
    令
    \begin{gather}
        X=\begin{bmatrix}
            x_{11} & x_{12}\\
            x_{21} & x_{22}
        \end{bmatrix}=\begin{bmatrix}
            a & a\\
            -a & -a
        \end{bmatrix},\qquad Y=\begin{bmatrix}
            y_1\\
            y_2
        \end{bmatrix}=\begin{bmatrix}
            b\\
            -b
        \end{bmatrix}
    \end{gather}
    其中,$a\neq0$,$b\neq0$。岭回归的参数估计应满足使目标函数 $Obj_{ridge}$ 达到最小。令 $Obj_{ridge}$ 分别对$\beta_0$、$\beta_1$和$\beta_2$求偏导,使偏导数等于零,得到
    \begin{equation}
        \begin{cases}
            \frac{\partial Obj_{ridge}}{\partial\beta_0}=-2\left(b-\beta_0-a\beta_1-a\beta_2\right)-2\left(b-\beta_0+a\beta_1+a\beta_2\right)=0\\
            \frac{\partial Obj_{ridge}}{\partial\beta_1}=-2a\left(b-\beta_0-a\beta_1-a\beta_2\right)+2a\left(b-\beta_0+a\beta_1+a\beta_2\right)+2\lambda\beta_1=0\\
            \frac{\partial Obj_{ridge}}{\partial\beta_2}=-2a\left(b-\beta_0-a\beta_1-a\beta_2\right)+2a\left(b-\beta_0+a\beta_1+a\beta_2\right)+2\lambda\beta_2=0
        \end{cases}
    \end{equation}
    解得
    \begin{equation}
        2\lambda\beta_1=2\lambda\beta_2
    \end{equation}
    因为 $\lambda\neq 0$,所以有 $\hat{\beta_1}=\hat{\beta_2}$。
    \subsubsection*{c}
    LASSO回归最优化问题的目标函数为
    \begin{equation}
        Obj_{LASSO}=\sum_{i=1}^2\left(y_i-\beta_0-\beta_1x_{i1}-\beta_2x_{i2}\right)^2+\lambda\left(\left|\beta_1\right|+\left|\beta_2\right|\right)
    \end{equation}
    \subsubsection*{d}
    LASSO回归的参数估计应满足使目标函数 $Obj_{LASSO}$ 达到最小。令 $Obj_{LASSO}$ 分别对$\beta_0$、$\beta_1$和$\beta_2$求偏导,使偏导数等于零,得到
    \begin{equation}
        \begin{cases}
            \frac{\partial Obj_{LASSO}}{\partial\beta_0}=-2\left(b-\beta_0-a\beta_1-a\beta_2\right)-2\left(b-\beta_0+a\beta_1+a\beta_2\right)=0\\
            \frac{\partial Obj_{LASSO}}{\partial\beta_1}=-2a\left(b-\beta_0-a\beta_1-a\beta_2\right)+2a\left(b-\beta_0+a\beta_1+a\beta_2\right)+\lambda\frac{\partial\left|\beta_1\right|}{\partial\beta_1}=0\\
            \frac{\partial Obj_{LASSO}}{\partial\beta_2}=-2a\left(b-\beta_0-a\beta_1-a\beta_2\right)+2a\left(b-\beta_0+a\beta_1+a\beta_2\right)+\lambda\frac{\partial\left|\beta_2\right|}{\partial\beta_2}=0
        \end{cases}
    \end{equation}
    解得
    \begin{equation}
        \lambda\frac{\partial\left|\beta_1\right|}{\partial\beta_1}=\lambda\frac{\partial\left|\beta_2\right|}{\partial\beta_2}
    \end{equation}
    因为 $\lambda\neq 0$,则有 $\frac{\partial\left|\beta_1\right|}{\partial\beta_1}=\frac{\partial\left|\beta_2\right|}{\partial\beta_2}$,其中
    \begin{equation}
        \frac{\partial\left|\beta_i\right|}{\partial\beta_i}=
        \begin{cases}
            1,&\beta_i>0\\
            -1,&\beta_i<0
        \end{cases}
        ,\qquad i=1,2
    \end{equation}
    即偏导数的值取决于 $\beta_1$ 和 $\beta_2$ 的正负号,因此 $\hat{\beta_1}$ 和 $\hat{\beta_2}$ 不唯一,且只需满足 $\hat{\beta_1}\hat{\beta_2}>0$($\hat{\beta_1}$ 和 $\hat{\beta_2}$ 同号)。
    \section{\heiti{第9题}}
    \vspace{10pt}
    \subsubsection*{a}
    将 College 数据集中的 Private 变量转化为哑变量,令1表示$Yes$,0表示$No$。按 3:1 的比例将数据集划分为训练集和测试集,其中训练集包含583行数据,测试集包含194行数据。
    \subsubsection*{b}
    通过逐步选择法筛选模型的解释变量,结果如下:
    \begin{figure}[H]
        \centering
        \includegraphics[scale=0.25]{Stepwise.png}
    \end{figure}
    选取显著性水平最高的5个解释变量 Accept、Top10perc、Expend、Outstate、 Enroll 建立线性回归模型,结果如下:
    \begin{figure}[H]
        \centering
        \includegraphics[scale=0.25]{Linear.png}
    \end{figure}
    其中模型参数均显著。用该模型对测试集数据进行预测,得到测试误差为1232.63。
    \subsubsection*{c}
    使用10折交叉验证法确定岭回归的最佳 $\lambda=368.2153$,建立岭回归模型,系数估计结果如下:
    \begin{figure}[H]
        \centering
        \includegraphics[scale=0.25]{Ridge.png}
    \end{figure}
    用该模型对测试集数据进行预测,得到测试误差为3763.05。
    \subsubsection*{d}
    使用10折交叉验证法确定LASSO回归的最佳 $\lambda=29.1804$,建立LASSO回归模型,系数估计结果如下:
    \begin{figure}[H]
        \centering
        \includegraphics[scale=0.25]{LASSO.png}
    \end{figure}
    用该模型对测试集数据进行预测,得到测试误差为1405.266。
\end{document}